% Vorlesung vom 17.11.2015
\renewcommand{\ldate}{2015-11-17}

\subsection{Beispiel Urne mit 2 Kugeln, n mal ziehen mit Rücklegen, 3 Fälle möglich}
\includegraphicsdeluxe{BspUrne2n31.jpg}{Beispiel Urne}{Urne mit 2 Kugeln, n mal ziehen mit Rücklegen, 3 Fälle möglich}{fig:BspUrne2n31}
Urne mit 2 Kugeln, n mal ziehen mit Rücklegen, 3 Fälle möglich (Abb. \ref{fig:BspUrne2n31}): 
\begin{enumerate}
\item $A_1$: 2 schwarze
\item $A_2$: 1 rote, 1 schwarze
\item $A_3$: 2 rote
\end{enumerate}
Man bekommt die Information B: Nur rote Kugeln wurden gezogen.
Berechne $P(A_1|B), P(A_2|B), P(A_3|B)$. 
Ergebnis: Alle Pfade

$P(A_1|B)$
$=0$

$P(A_2|B)$
$=\frac{P(A_2\cap B)}{P(B)}$
$=\frac{p_2 \cdot \rbr{\frac{1}{2}}^n}{p_2 \cdot \rbr{\frac{1}{2}}^n + p_3}$
$\overrightarrow{n\rightarrow \infty} 0$

$P(A_3|B)$
$=\frac{P(A_3\cap B)}{P(B)}$
$=\frac{p_3}{p_2 \cdot \rbr{\frac{1}{2}}^n + p_3}$
$\overrightarrow{n\rightarrow \infty} 1$

\paragraph{Mit der Formel von Bayes}
$P(A_k|B)$
$=\frac{P(A_k) \cdot P(B|A_k)}{\sum_{j=1}^{3} P(A_j) \cdot P(B|A_j)}$

Nenner: $p_1 \cdot 0 + p_2 \cdot \rbr{\frac{1}{2}}^n + p_3 \cdot 1 $
$=p_2 \cdot \rbr{\frac{1}{2}}^n + p_3 $

Zähler: $k=1: 0$, $k=2: p_2 \cdot rbr{\frac{1}{2}}^n$, $k=3: p_3 \cdot 1$

$P(A_1|B)$
$=0$

$P(A_2|B)$
$=\frac{p_2 \cdot \rbr{\frac{1}{2}}^n}{p_2 \cdot \rbr{\frac{1}{2}}^n + p_3}$

$P(A_3|B)$
$=\frac{p_3}{p_2 \cdot \rbr{\frac{1}{2}}^n + p_3}$

\subsection{Beispiel Würfeln}
\includegraphicsdeluxe{BspWueSechser1.jpg}{Beispiel Würfeln:}{Beispiel Würfeln: B: Augensumme $\geq 8$, $A_1$: kein Sechser, $A_2$: mindestens ein Sechser}{fig:BspWueSechser1}
Es werden zwei Würfel geworfen. Wir bekommen die Information: Augensumme $\geq 8$. Wie groß ist die Wahrscheinlichkeit, dass wir mindestens einen Sechser haben (Abb. \ref{fig:BspWueSechser1})?

Exakt: Es muss vor dem Würfeln ausgemacht werden: Man bekommt die Information Augenzumme $\geq 8$ \underline{oder} Augensumme $< 8$. 

$P(A_2|B)$
$=\frac{P(A_2\cap B)}{P(B)}$
$=\frac{\frac{9}{36}}{\frac{15}{36}}$
$=\frac{9}{15}$
$=\frac{3}{5}$

$P(B)$
$=\frac{5}{36} + \frac{4}{36} + \frac{3}{36} + \frac{2}{36} + \frac{1}{36} $
$=\frac{15}{36}$

\subsection{Beispiel Test auf Krankheit}
\includegraphicsdeluxe{BspTestKrankheit1.jpg}{Beispiel Test auf Krankheit}{Beispiel Test auf Krankheit: A: Person krank, B: Test zeigt positiv}{fig:BspTestKrankheit1}
Test positiv $\Rightarrow$ krank\\
Test negativ $\Rightarrow$ gesund\\
$P_{sl}$ Wahrscheinlichkeit(Test zeig positiv | Person krank)\\
$P_{sp}$ Wahrscheinlichkeit(Test zeig negativ | Person gesund)\\

ELISA-Test auf HIV: $P_{sl} = P_{sp} = 0.998$

Person wird getestet. Test zeigt positiv. Wie groß ist die Wahrscheinlichkeit, dass die Person krank ist? q sei die apriori Wahrscheinlichkeit, dass die Person krank ist ($q \cdot 100 \%$ der Bevölkerung hat HIV, vgl. Abb. \ref{fig:BspTestKrankheit1})

$P(A|B)$
$=\frac{A\cap B}{P(B)}$
$=\frac{q \cdot P_{sl}}{q \cdot P_{sl} + (1-q) (1 - P_{sp}}$

Hier: 
$P(A|B)$
$=\frac{q \cdot 0.998}{q \cdot 0.998 + (1-q) \cdot 0.002}$

\begin{tabular}{|c|c|c|c|}
\hline q & 0.001 & 0.01 & 0.1 \\ 
\hline $P(A|B)$ & 0.333 & 0.834 & 0.982 \\ 
\hline 
\end{tabular} 

\subsection{Verblüffende Beispiele}

\paragraph{1)}
Eine Familie hat 2 Kinder. Man bekommt die Information \glqq Mindestens ein Junge \grqq. Wie groß ist die Wahrscheinlichkeit, dass es 2 Jungen sind?

A: 2 Jungen\\
B: mindestens ein Junge\\
$P(A|B)$
$=\frac{P(A\cap B)}{P(B)}$
$=\frac{\frac{1}{4}}{\frac{3}{4}}$
$=\frac{1\cdot 4}{4\cdot 3}$
$=\frac{1}{3}$

\paragraph{2)}
\includegraphicsdeluxe{BspJungenMaedchen2.jpg}{Beispiel Familie 2}{Beispiel Familie mit zwei Kindern}{fig:BspJungenMaedchen2}
Familie mit 2 Kindern. Die 2 Kinder spielen im Haus. Eines schaut aus dem Fenster heraus. Es ist ein Junge. Wie groß ist die Wahrscheinlichkeit, dass das andere Kind auch ein Junge ist? 

$P(\textrm{JJ} | \textrm{Junge schaut heraus})$
$=\frac{P(JJ \cap J_{sh})}{P(J_{sh})}$
$=\frac{\frac{1}{4}}{\frac{1}{4} + \frac{1}{2} \cdot \frac{1}{2}}$
$=\frac{\frac{1}{4}}{\frac{1}{2}}$
$=\frac{1}{2}$

Mädchen sind nun neugieriger als Jungen (vgl. rote Wahrscheinlichkeiten in Abb. \ref{fig:BspJungenMaedchen2}).
$ P(JJ|J_{sF})$
$=\frac{P(JJ\cap J_{sF})}{P(J_{sF})}$
$=\frac{\frac{1}{4}}{\frac{1}{4} + \frac{1}{2} \cdot \frac{1}{4}}$
$=\frac{\frac{1}{4}}{\frac{3}{8}}$
$=\frac{1\cdot 8}{4 \cdot 3}$
$=\frac{2}{3}$

\section{Stochastische Unabhängigkeit}
$P(A|B)$
$=\frac{A\cap B}{P(B)}$

$P(A|B)$
$=P(A) \Rightarrow$ A ist von B unabhängig. 

$\frac{P(A\cap B)}{P(B)}$
$=P(A) $
$\Leftrightarrow$
$P(A\cap B)$
$=P(A) \cdot P(B)$

\begin{defi}
Zwei Ereignisse A, B heißen unabhängig, wenn gilt: $P(A\cap B) = P(A) \cdot P(B)$
\end{defi}

\begin{defi}
Drei Ereignisse A,B,C heißen unabhängig, wenn gilt: \\
$P(A\cap B\cap C) = P(A) \cdot P(B) \cdot P(C)$\\
$P(A\cap B) = P(A) \cdot P(B)$\\
$P(A\cap C) = P(A) \cdot P(C)$\\
$P(B\cap C) = P(B) \cdot P(C)$
\end{defi}

\begin{defi}
$A_1, ..., A_n$ heißen unabhängig, wenn: 
$P(\bigcap_{j\in T} A_j)$
$=\prod_{j\in T} P(A_j)$
für jede Teilmenge T aus $\cbr{1,2,...,n}$.
\end{defi}