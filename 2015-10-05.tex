% Vorlesung vom 05.10.2015
\renewcommand{\ldate}{2015-10-05}

\section{Allgemeines}

\begin{itemize}
\item kein Skript. Alles wichtige steht an der Tafel.
\item Literatur: Norbert Henze, Stochastik für Einsteiger, Vieweg/Teubner
\item keine Hausaufgaben
\item Übungsaufgaben gibt es immer zwischendurch und alte Prüfungsaufgaben gegen Ende des Semesters
\end{itemize}

\section{Zufallsexperimente}

Experimente: Würfeln, Ergebnismenge $ \Omega = \{1,2,3,4,5,6\}$, Münze werfen: $ \Omega = \{K, Z\}$. Solange würfeln bis eine 6 kommt: $ \Omega = \{ 1,2,3,4,...\} = \N $\\
\profnote{Man kann mehrere Exerimente auch zusammenfassen}n Einzelexperimente $ \Rightarrow $ Kartesisches Produkt: $ \Omega_1, \Omega_2, ...,  \Omega_n, \Omega = \Omega_1 \times \Omega_2 \times ... \times  \Omega_n $\\
Zum Beispiel erst würfeln, dann Münze: \\
$ \Omega_1 = \{1,...,6\}, \Omega_2=\{K,Z\} \Rightarrow \Omega=\{(a,b) | a\in \Omega_1, b\in \Omega_2\}$\\
Zweimal nacheinander würfeln: $ \underbrace{\Omega}_{1. Wurf}=\{(\underbrace{a}_{1.},\underbrace{b}_{2. Wurf}) : 1 \leq a,b \leq 6 \} $\\

Mit rotem und grünen Würfel gleichzeitig: $ \Omega = \{(\underbrace{a}_{gr"uner}, \underbrace{b}_{roter}) | 1\leq a,b\leq 6\} $\\
In einer Urne sind die Kugeln 1 bis n. Es wird k mal mit Rücklegen gezogen: $ \Omega = \{(\underbrace{a_1}_{1. Zug}, \underbrace{a_2}_{2. Zug}, ..., a_k) | 1\leq a_i \leq n \}$

In einer Schachtel mit den Kugeln 1, 2, 3, 4 werden zwei mit einem Griff gezogen. Wie sieht $ \Omega $ aus?\\
$ \Omega=\{\{a_1,a_2\} | a_1 \neq a_2, 1\leq a_1,a_2\leq 4\} $\\
$ \Omega=\{\{a_1,a_2\} | 1\leq a_1, a_2\leq 4, a_1 < a_2 \} $\\

Lotto: 6 Kugeln aus 49: $ \Omega=\{(\underbrace{a_1}_{1. Kugel}, \underbrace{a_2}_{2. Kugel}, a_3, a_4, a_5, a_6) | a_i \textrm{sind verschieden} \}$ (mit Reihenfolge)\\
$ \Omega=\{(a_1,...,a_6) | a_1 < a_2 < a_3 < a_4 < a_5 < a_6\}$ (ohne Reihenfolge)\\

\subsection{2. Ereignisse}
$ A \subset \Omega$ heißt Ereignis, Ergebnis $ \omega $. A ist eingetreten, wenn $\omega \in A$. \\
$\{\omega\} $: Elementarereignis\\
$ \Omega $: sicheres Ereignis\\
$\emptyset $: unmögliches Ereignis\\
$A \cap B$: A und B sind eingetreten.\\
$A\cup B$: A oder B\\
$A_1\cap A_2\cap ... \cap A_n$\\
$A_1\cup A_2\cup ... \cup A_n$\\
$B \setminus A= \{\omega \in B : \omega \notin A \} $: B minus A. \\
Gegenseitiges Komplement: $ \overline{A} = \Omega \setminus A = \{\omega : \omega \notin A \}$\\
A, B heißen disjunkt, wenn $ A \cap B = \emptyset $. Für $ A \cup B $ schreibt man dann $A+B$.\\

\subsection{Zweimal Würfeln}
A: erster Wurf 5 $\Rightarrow A=\{(5,1),(5,2),...(5,6)\}$\\
B: zweiter Wurf höher als erster 
$\Rightarrow B = \{ (1,2), (1,3), (1,4), (1,5), (1,6), (2,3), (2,4), $
$(2,5), (2,6), (3,4), (3,5), (3,6), (4,5), (4,6), (5,6) \} $

\subsection{Rechenregeln in der Mengenlehre}
Kommutativ: $ A\cap B = B\cap A $ und $ A\cup B = B\cup A $\\
Assoziativ: $ A\cap (B\cap C) = (A\cap B) \cap C$ und $ A\cup (B\cup C) = (A\cup B) \cup C$ \\
De Morgan: $ \overline{(A\cup B)} = \bar{A}\cap \bar{B} $ und $ \overline{(A\cap B)} = \bar{A} \cup \bar{B}$, Beispiel: $ |A\cup B| = |A|+|B|-|A\cap B|$ \\
Distributiv: $ A \cap (B\cup C) = (A\cap B)\cup (A\cap C)$ und $ \underbrace{A\cup (B\cap C)}_{L} = \underbrace{(A\cup B) \cap (A\cup C)}_{R}$

\paragraph{Beweis $ L = R $}
\begin{enumerate}
\item $x \in L \Rightarrow x\in R \Rightarrow L \subset R $
\item $x \in R \Rightarrow x\in L \Rightarrow R \subset L $
\end{enumerate}
$\Rightarrow R=L \Box$\\

\subsection{Zufallsvariable}

Abbildung: $ X: \Omega \rightarrow \R$ heißt Zufallsvariable, z.B. zweimal würfeln. X ist die Augensumme: $ X(\omega_1, \omega_2) = \omega_1+\omega_2$\\
$ X_1$: erste Augenzahl\\
$ X_2$: zweite Augenzahl\\
$X = X_1+X_2$\\
$(X_1 + X_2)(\omega) = X_1(\omega) + X_2(\omega) = \omega_1 + \omega_2$\\

\paragraph{Abkürzungen}
$ \{X=K\} = \{\omega\in\Omega | X(\omega) = K\}$\\
Augensumme mindestens 10: $ \{X \geq 10\} $\\
Augensumme zwischen 3 und 8: $ \{3 \leq X \leq 8\}$\\
Wertebereich von $X$ ist $X(\Omega) = \{ X(\omega) : \omega \in \Omega \} $ \\
Arithmetik mit Zufallsvariable (ZV): \\
$ (X\cdot Y)(\omega) = X(\omega) \cdot Y(\omega)$\\
$ (a \cdot X)(\omega) = a \cdot X(\omega)$\\
$ \{X\leq Y\} = \{\omega\in\Omega | X(\omega) \leq Y(\omega) \}$\\
 
\subsection{Indikatorfunktionen}
$ I_{A}(\omega) = $ 1, falls $ \omega\in A $, sonst 0 ($A\in \Omega$). Die Indikatorfunktion I zeigt an, ob das Ereignis eingetreten ist oder nicht. \\
$ I_{A\cap B} = I_A \cdot I_B $\\
$ I_A = I\{A\} $ sind mögliche Schreibweisen im Buch.\\ 
$ I_{A_1\cap A_2\cap ... \cap A_n} = I_{A_1}\cdot I_{A_2}\cdot ... \cdot I_{A_n} $\\
$ I_{\bar{A}} = 1 - I_A $\\
$ I_A = I_A \cdot I_A$

\subsection{Zählvariable}

$ A_1, ..., A_n $ Ereignisse: $ X=I_{A_1} + I_{A_2} + ... + I_{A_n} $. X zählt, wie viele Ereignisse $A_1, ..., A_n$ eingetreten sind. \\
$ X(\omega) = $ Anzahl der $A_i$, in denen $\omega$ liegt. $ \{X=n\} = A_1\cap A_2\cap ...\cap A_n$ und $ \{X=0\} = \bar{A_1} \cap \bar{A_2} \cap ... \cap \bar{A_n} $

\paragraph{Beispiel:} Ein Treffer-Niete-Experiment wird n-mal wiederholt.\profnote{Treffer=1, Niete=0} \\
$ \Omega=\{(a_1,a_2,...,a_n) : a_i=1/0 \} $\\
$ A_j = \{\omega : \omega_j=1\} $ in j-ten Versuch Treffer\\
$ X=I_{A_1} + I_{A_2} + ... + I_{A_n} $ zählt Anzahl der Treffer\\
$ X(\omega) = X(\omega_1, \omega_2, ..., \omega_n) = \omega_1 + \omega_2 + ... + \omega_n $

\paragraph{Übung 3.5 (Buch)}
$ \Omega = \{ ( \omega_1, \omega_2, \omega_3 ) | 1 \leq \omega_i \leq 6 \} $\\
$ X: \Omega \rightarrow \R $\\
$ (\omega_1, \omega_2, \omega_3) \rightarrow $ 100, wenn $\omega_1=6$\\
$ (\omega_1, \omega_2, \omega_3) \rightarrow $ 50, falls $\omega_2=6$ und $\omega_1 \neq 6$\\
$ (\omega_1, \omega_2, \omega_3) \rightarrow $ 10, falls $\omega_2 = 6$ und $\omega_a, \omega_2\neq 6$\\
$ (\omega_1, \omega_2, \omega_3) \rightarrow $ -30, falls $\omega_1, \omega_2, \omega_3\neq 6$\\
