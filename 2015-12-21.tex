% Vorlesung vom 21.12.2015
\renewcommand{\ldate}{2015-12-21}

\section{Diskrete Wahrscheinlichkeitsräume}

$\Omega$ abzählbar unendlich.  $(\Omega, P)$ diskreter Wahrscheinlichkeitsraum, wenn: P auf den Teilmengen von $\Omega$ definiert mit:

\begin{enumerate}
\item $P(A) \geq 0 $
\item $P(\Omega) = 1$
\item $P(\sum_{i=1}^{\infty} A_i) = \sum_{i=1}^{\infty} P(A_i)$ disjunkte $A_i$
\end{enumerate}

Die $\varphi$ Additivität ($\sum_{i=1}^{\infty} A_i$) folgt nicht aus $P(A\cup B) = P(A) + P(B)$ für $A\cap B = \emptyset$

\subsection{Beispiel}
$\Omega = \cbr{\omega_1, \omega_2, \omega_3, ..., }$

$P(\omega_j) \geq \forall j$, 
$\sum_{j=1}^{\infty} p(\omega_j) = 1$,
$P(A) = \sum_{\omega_j \in A} p(\omega_j)$ ist ein Wahrscheinlichkeitsmaß.

$E(X) = \sum_{\omega \in \Omega} X(\omega) \cdot p(\omega)$

$E(X) = \sum_{j=1}^{\infty} x_j \cdot P(X=x_j)$

$E(g(X)) = \sum_{j=1}^{\infty} g(x_j) \cdot P(X=x_j)$

\subsection{St. Petersburger Spiel}
Eine Münze wird solange geworfen bis eine Zahl oben ist. 

$\Omega = \cbr{1,2,3,4,...} \Rightarrow P(\cbr{k}) = \rbr{\frac{1}{2}}^k$.
Man bekommt $2^{k-1}$ Rubl als Gewinn. 
X ist der Gewinn.

$
EX = \sum_{k=1}^{\infty} p(k) \cdot X(k) 
= \sum_{k=1}^{\infty} \rbr{\frac{1}{2}}^k \cdot 2^{k-1}
= \sum_{k=1}^{\infty} \frac{2^{k-1}}{2^k}
= \sum_{k=1}^{\infty} \frac{1}{2} 
= \frac{1}{2} + \frac{1}{2} + \frac{1}{2} + ... = \infty
$.  
Erwartungswert ist $\infty$.

\subsection{Spieler-Ruin-Problem}
Der Spieler A hat ein Kapital von a Euro und der Spieler B eins von b. Sie werfen eine Münze. Gewinnt A, so bekommt A von B einen Euro und umgekehrt. Sie spielen solange, bis einer bankrott ist. 

\textbf{Wahrscheinlichkeit Allgemeiner:}

A gewinnt: p 

B gewinnt: (1-p) = q

\textbf{Wie groß ist die Wahrscheinlichkeit, dass A gewinnt?}

\includegraphicsdeluxe{WahrschlABPfade1.jpg}{Spieler-Ruin-Problem}{Spieler-Ruin-Problem: $\Omega$: alle möglichen Pfade. }{fig:WahrschlABPfade1}
Wir rechnen das allgemein aus (Abb. \ref{fig:WahrschlABPfade1}), ges.: 

$P_k(A)$ Kapital von A\\
$P_0(A) = 0$\\
$P_r(A) = 1$\\
Sei $1\leq k\leq r-1$\\
$P_k(A) = P_k(A\cap A_1) + P_k(A\cap A_2)$, wobei $A_1:$ gewinnt beim nächsten Wurf, $A_2$ verliert ((Abb. \ref{fig:})).
$
= P_k(A_1) \cdot P_k(A|A_1) + P_k(A_2) \cdot P_k(A|A_2)
= p \cdot P_{k+1}(A) + q \cdot P_{k-1}(A)
$

$d_k = P_{k+1}(A) - P_k(A)$

$d_{k-1} = P_{k}(A) - P_{k-1}(A)$

\profnote{Wir lassen das von A also (A) mal weg $\Rightarrow$ weniger Schreibarbeit. }

$
\frac{d_k}{d_{k-1}} = \frac{P_{k+1} - P_{k+1}\cdot p - P_{k-1} \cdot q}{P_{k+1} \cdot p + P_{k-1} \cdot q - P_{k-1}}
= \frac{P_{k+1} \underbrace{(1-p)}_{q} - P_{k-1} \cdot q}{P_{k+1} \cdot p + P_{k-1} \underbrace{(q-1)}_{-p}}
= \frac{P_{k+1} \cdot q - P_{k-1} \cdot q}{P_{k+1} \cdot - P_{k-1} \cdot p}
= \frac{q \rbr{P_{k+1} - P_{k-1}}}{p \rbr{P_{k+1} - P_{k-1}}}
= \frac{q}{p}
\Rightarrow d_k = d_{k-1} \cdot \frac{q}{p}
\Rightarrow P_{k+1} = P_k + d_k
$

$P_1 = P_0 + d_0 = d_0$

$P_2 = P_1 + d_1 = d_0 + d_1$

$\vdots$

$P_k = d_0 + d_1 + ... + d_{k-1}$

$p_r = 1$\\

\paragraph{1. Fall:} $p=q=\frac{1}{2}$. Alle $d_k$ gleich.

$1=d_0 + d_1 + ... + d_{r-1} \Rightarrow d=\frac{1}{r}, d = d_0 = d_1 = ... $

$\Rightarrow P_k = k \cdot \frac{1}{r} = \frac{k}{r}$. Also $P_a = \frac{a}{a+b}$

\textbf{Beispiel} A hat 10 Euro und B hat 12. $P(A) = \frac{10}{10+12} = 0.45$

\paragraph{2. Fall:} $p\neq q$

$d_1 = d_0 \cdot \frac{1}{p}$

$d_2 = d_1 \cdot \frac{q}{p}
= d_0 \cdot \rbr{\frac{q}{p}}^2
\Rightarrow P_k(A) = \sum_{j=0}^{k-1} d_j
= \sum_{j=0}^{k-1} d_0 \cdot \rbr{\frac{q}{p}}^j
= d_0 \sum_{j=0}^{k-1} \rbr{\frac{q}{p}}^j
= d_0 \frac{1 - \rbr{\frac{q}{p}}^k}{1 - \frac{q}{p}}
= P_k(A)
$

Für $k=r$ ergibt sich 
$P_r(A) = 1$

$ 
1 = d_0 \cdot \frac{1 - \rbr{\frac{q}{p}}^r}{1 - \frac{q}{p}}
\Rightarrow d_0 = \frac{1 - \rbr{ \frac{q}{p} } }{1 - \rbr{ \frac{q}{p} }^r }
\Rightarrow P_k(A) = \frac{1 - \frac{q}{p} }{1 - \rbr{ \frac{q}{p} }^r } \cdot \frac{1 - \rbr{\frac{q}{p}}^k }{1 - \rbr{ \frac{q}{p} } }
\Rightarrow P_k(A) = \frac{1 - \rbr{\frac{q}{p}}^k }{1 - \rbr{ \frac{q}{p} }^r }
$

\paragraph{Beispiel}
$ a=10, b=12, p=0.6, q=0.4 \Rightarrow P(A) = \frac{1 - \rbr{\frac{4}{6}}^{10} }{1-\rbr{\frac{4}{6}}^{22} } = 0.98$
 
\section{Exkurs wichtige Reihen}
$
e^x 
= \sum_{k=0}^{\infty} \frac{x^k}{k!}
$
Exp.reihe

$\sum_{k=0}^{\infty} x^k = \frac{1}{1-x}$ für $\abs{1} < 1$ geometrische Reihe. 

Ableitung: 
$\sum_{k=1}^{\infty} k\cdot x^{k-1} = \frac{1}{(1-x)^2}$

Ableitung: 
$\sum_{k=2}^{\infty} k (k-1) \cdot x^{k-2} = \frac{2}{(1-x)^3}$

\begin{satz}[Umordnungssatz]
$a_i \geq 0, \forall i\in \N$

$\sum_{n=1}^{\infty} a_n < \infty$

Dann darf mann: 
\begin{enumerate}
\item Reihe beliebig umordnen $\sum_{n=1}^{\infty} a_{\pi(n)}$ ($\pi$ Permutationen)
\item In Teilmengen $T_1, T_2, ... $ zerlegen und aufaddieren $\sum_{i} \rbr{ \sum_{a_k \in T_i} a_k }$
\end{enumerate}
\end{satz}