% Vorlesung vom 03.11.2015
\renewcommand{\ldate}{2015-11-03}

\subsection{Sortieralgorithmus}
Wir haben die verschiedenen Zahlen $a_1, a_2, ..., a_n$. Wir betrachten die Permutationen dieser Zahlen. Eine Permutation $(a_1, ..., a_n)$ soll sortiert sortiert werden (von klein nach groß). Vorgehen:\\

\begin{itemize}
\item $a_1$ lassen
\item Ist $a_2$ kleiner als $a_1$, dann vertauschen.
\item Dann $a_3$ durch Vertauschen einsortieren. 
\item $\vdots$
\item bis $a_n$
\end{itemize}

\subsection{Beispiel}
(13,10,15,11) 
\begin{enumerate}
\item (10,13,15,11)
\item (10,13,11,15)
\item (10,11,13,15), fertig!
\end{enumerate}

Die Zufallsgröße X zählt die Anzahl der Vertauschungen. 
$X(13,10,15,11) = 3$

\paragraph{Wir wollen EX berechnen}
$Y_j$ Einsortieren von $a_j$ (Anzahl der Vertauschungen), $2\leq j \leq n$\\
$Y_j(a_1,...,a_j,...,a_n) = $ \profnote{Wieviele Werte vor $a_j$ sind größer als $a_j$? So viele muss ich vertauschen.}
Anzahl der $a_i$ mit $a_i < a_i, i < j $. Das kann man auch so ausrechnen: 
$Y_j = \sum_{i=1}^{j-1} I\cbr{a_j < a_i}$ \profnote{Die Indikatorfunktion I zählt auch die Anzahl der Vertauschungen.}

$X(a_1,...,a_n)$
$=\sum_{j=2}^{n} Y_j \cdot (a_1, ..., a_n)$ \profnote{Die Wahrscheinlichkeit der Indikatorfunktion ist immer die Wahrscheinlichkeit der Ereignisse $a_i$}

Wir brauchen $P(a_j < a_i)$ ist $\frac{1}{2}$ (plausibel), also:\\

$ E Y_j = \sum_{i=1}^{j-1} P(a_j < a_i)$
$= \sum_{i=1}^{j-1} \frac{1}{2} $
$= (j-1) \cdot \frac{1}{2}$

$\Rightarrow E X $
$=\sum_{j=2}^{n} E Y_j $
$= \sum_{j=2}^{n} (j-1) \cdot \frac{1}{2}$
$=\frac{1}{2} \sum_{j=1}^{n-1} j $
$= \frac{1}{2} \frac{(n-1) \cdot n}{2}$
$= \frac{n^2 - n}{4}$. 
Der Erwartungswert EX wächst quadratisch mit n. Für $n=10 \Rightarrow EX=\frac{100-10}{4} = 22.5$, 
$n=30 \Rightarrow EX = 217.5$

\subsection{Transformationsformel}
% 1 \includegraphicsdeluxe{transformationsformel1.jpg}{Transformation}{Transformation}{fig:transformationsformel1}
Wir betrachten die Zufallsgröße $g\circ X, \sbr{(g\circ X)(\omega) = g(X(\omega))}$. 
$x_1, ..., x_k$ ist der Wertebereich von X.

$E(g\circ X) $
$=\sum_{j=1}^{k} g(x_j) \cdot P(X=x_j)$. Speziell für $g=id$ folgt: 
$ EX$
$= \sum_{j=1}^{k} x_j P(X=x_j)$ (schon bekannt).

\begin{proof}
% 2 \includegraphicsdeluxe{beweisTransFormel1.jpg}{Zerlegung in Teilmengen}{$\Omega$ wird in die Teilmengen $A_1, A_2, ..., A_k$ zerlegt}{fig:beweisTransFormel1}
Zerlegung von $\Omega$: $A_j = \cbr{\omega \in \Omega: x(\omega) = x_j}$

$E(g(X)) $
$= \sum_{\omega \in \Omega} g(X(\omega)) \cdot p(\omega)$
$= \sum_{j=1}^{k} \sum_{\omega \in A_j} g(X(\omega)) \cdot p(\omega)$
$=\sum_{j=1}^{k} g(x_j) \cdot \sum_{\omega \in a_j} p(\omega)$
$=\sum_{j=1}^{k} g(x_j) \cdot P(A_j)$
$= \sum_{j=1}^{k} g(x_j) \cdot P(X=x_j)$
\end{proof}

\subsection{Beispiel würfeln mit zwei Würfeln}
X: größere Augenzahl, $g(x) = x^2$, gesucht: $E(g(X))$

$E(g\circ X)$
$=1^2 \cdot \frac{1}{36} $
$+ 2^2 \cdot \frac{3}{36} $
$+ 3^2 \cdot \frac{5}{36} $
$+ 4^2 \cdot \frac{7}{36} $
$+ 5^2 \cdot \frac{9}{36} $
$+ 6^2 \cdot \frac{11}{36} $
$= ... $
$=22.0$

\section{Stichprobenentnahme}
Urne mit r roten und s schwarzen Kugeln. Die roten Kugeln sind nummeriert mit $1,2,...,r$ und die schwarzen Kugeln mit $r+1,...,r+s$. Es werden nacheinander (Stichprobe) n Kugeln \underline{ohne Rücklegen} gezogen. Der mögliche Grundraum ist 
$\Omega = \cbr{(a_1,...,a_n) : a_i \textrm{ verschieden}}$. Das Ereignis $A_j$ bedeutet jede j-te Kugel ist rot, also muss $a_j \leq r$.\\

$P(A_j) $
$= \frac{r}{r+s}$ ist plausibel. Wir machen trotzdem eine formale Rechnung: \profnote{Wir machen das mit der Abzählregel.}\\
\textbf{alle Fälle:} $ (r+s)^{\underline{n}} $ \profnote{Beispiel: $ 7^{\underline{3}} = 7\cdot 6\cdot 5$}\\
\textbf{günstige Fälle:} Kugel 1 an Stelle j: $(r+s-1)^{\underline{n-1}}$\\
$\vdots$\\
Kugel r an Stelle j: $(r+s-1)^{\underline{n-1}}$\\
$P(A_j)$
$=\frac{r \cdot (r+s-1)^{\underline{n-1}} }{(r+s)^{\underline{n}}}$
$=\frac{r\cdot (r+s-1) \cdot (r+s-2) \cdot ...  \cdot  (r+s-n+1)}{(r+s)(r+s-1) \cdot ... \cdot (r+s-n+1)}$
$=\frac{r}{r+s}$

\begin{satz}
r rote, s schwarze Kugeln. n Stück ziehen ohne Rücklegen.\\
$X(\omega) = $ Anzahl der roten Kugeln in der Ziehung. 
Die Verteilung von X heißt hypergeometrische Verteilung. 
Es gilt:
\begin{enumerate}
\item $ E X = n \cdot \frac{r}{r+s}$
\item $ P(X=k) = \frac{\binom r k \binom s {n-k}}{\binom {r+s} n}$
\end{enumerate}
\end{satz}

\begin{proof}
\begin{enumerate}
\item $X = \sum_{j=1}^{n} I_{A_j}$
$ E X = \sum_{j=1}^{n} E I_{A_j} $
$= \sum_{j=1}^{n} P(A_j)$
$= \sum_{j=1}^{n} \frac{r}{r+s}$
$= n \cdot \frac{r}{r+s}$
\item Modell wechseln! $\Omega = $ Alle Teilmengen mit genau n Kugeln.\\
\textbf{Alle Fälle:} $\binom{r+s}{n}$\\
\textbf{Günstige Fälle:} genau k rote Kugeln $\binom r k \cdot \binom{s}{n-k}$, also: 
$P(X=k) = \frac{\binom{r}{k} \cdot \binom{s}{n-k}}{\binom{r+s}{n}}$
\end{enumerate}

\end{proof}