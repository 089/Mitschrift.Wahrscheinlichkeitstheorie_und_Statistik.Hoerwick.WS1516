% Vorlesung vom 24.11.2015
\renewcommand{\ldate}{2015-11-24}

\subsection{Beispiel 2 mal würfeln}
X: Augenzahl im 1. Wurf, Y: maximale Augenzahl, $\Omega: \cbr{(i,j):1\leq i,j \leq 6}$. 
$(X,Y)(3,1)$
$=(3,3) $

\begin{tabular}{|c|c|c|c|c|c|c|c|}
\hline X/Y & 1 & 2 & 3 & 4 & 5 & 6 & $\sum$\\ 
\hline 1 & $ \frac{1}{36} $ & $ \frac{1}{36} $ & $ \frac{1}{36} $ & $ \frac{1}{36} $ & $ \frac{1}{36} $ & $ \frac{1}{36} $ & $ \frac{6}{36}$\\ 
\hline 2 & 0 & $ \frac{2}{36} $ & $ \frac{1}{36} $ & $ \frac{1}{36} $ & $ \frac{1}{36} $ & $ \frac{1}{36} $ & $ \frac{6}{36}$ \\ 
\hline 3 & 0 & 0 & $ \frac{3}{36} $ & $ \frac{1}{36} $ & $ \frac{1}{36} $ & $ \frac{1}{36} $ & $ \frac{6}{36}$ \\ 
\hline 4 & 0 & 0 & 0 & $ \frac{4}{36} $ & $ \frac{1}{36} $ & $ \frac{1}{36} $ & $ \frac{6}{36}$ \\ 
\hline 5 & 0 & 0 & 0 & 0 & $ \frac{5}{36} $ & $ \frac{1}{36} $ & $ \frac{6}{36}$ \\ 
\hline 6 & 0 & 0 & 0 & 0 & 0 & $ \frac{6}{36} $ & $ \frac{6}{36}$ \\ 
\hline $\sum$ & $ \frac{1}{36}$ & $ \frac{3}{36}$ & $ \frac{5}{36}$ & $ \frac{7}{36}$ & $ \frac{9}{36}$ & $ \frac{11}{36}$ & \textbf{1} \\
\hline 
\end{tabular} 


\subsection{Funktionen von Zufallsvariablen/-größen}
% 1 \includegraphicsdeluxe{.jpg}{Funktionen von Zufallsvariablen/-größen}{Funktionen von Zufallsvariablen/-größen: $P(g\circ (X,Y) = u) = P\cbr{\omega\in\Omega : g(X(\omega),Y(\omega))=u}}{fig:}

$P(g\circ (X,Y) = u) $
$= P\cbr{\omega\in\Omega : g(X(\omega),Y(\omega))=u} $

$=\sum_{i=1}^{r} \sum_{j=1}^{s} P(\omega\in\Omega: X(\omega)=x_i \und Y(\omega)=y_j)$
$=g(x_i,y_j)$
$=u$

$=\sum_{i=1}^{r} \sum_{j=1}^{s} P(X=x_i \und Y=y_j)$
$=g(x_i,y_j)$
$=u$

$=\sum_{i=1}^{r} \sum_{j=1}^{s} P^{(X),Y} (x_i,y_j)$
$=g(x_i,y_j)$
$=u$

Für die Verteilung $g\circ (X,Y)$ braucht man also nur $P^{(X,Y)}$ auf $\R^2$. Wir berechnen den Erwartungswert von $g\circ (X,Y)$. Es gibt drei Möglichkeiten diesen auszurechnen:
\begin{enumerate}
\item $E[g\circ (X,Y)] = \sum_{\oiO} g\circ(X,Y)(\omega)\cdot p(\omega)$
\item $E[g\circ (X,Y)] = \sum_{i=1,...,r\\j=1,...,s} P^{(X,Y)} (x_i,y_j) \cdot g(x_i,y_j)$
\item $E[g\circ (X,Y)] = \sum_{u\in \textrm{ Wertebereich von } g\circ (X,Y)} u\cdot P^{g\circ(X,Y)}$
\end{enumerate}

\subsection{Beispiel 2 mal würfeln}
X: erste Augenzahl, Y: maximale Augenzahl, Verteilung und Erwartungswert von $X\cdot Y=Z, g(x,y)=x\cdot y$\\

\textbf{Mit Verfahren 2)}\\
$EZ = 1\cdot \frac{1}{36} + 2\cdot \frac{1}{36} + 3\cdot \frac{1}{36} + 4\cdot \frac{1}{36} + 5\cdot \frac{1}{36} + 6\cdot \frac{1}{36} + 4\cdot \frac{2}{36} + 6\cdot \frac{1}{36} + 8\cdot \frac{1}{36} + 10\cdot \frac{1}{36} + 12\cdot \frac{1}{36} + 9\cdot \frac{3}{36} + 12\cdot \frac{1}{36} + 15\cdot \frac{1}{36} + 18\cdot \frac{1}{36} + 16\cdot \frac{4}{36} + 20\cdot \frac{1}{36} + 24\cdot \frac{1}{36} + 25\cdot \frac{5}{36} + 30\cdot \frac{1}{36} + 36\cdot \frac{6}{36} = \frac{616}{36} = 17.11$\\

\textbf{Mit Verfahren 3)}\\  
\begin{tabular}{|c|c|c|c|c|c|c|c|c|c|}
\hline Wertebereich von Z & 1 & 2 & 3 & 4 & 5 & 6 & 8 & 9 & ... \\ 
\hline Verteilung von Z & $\frac{1}{36}$ & $\frac{1}{36}$ & $\frac{1}{36}$ & $\frac{3}{36}$ & $\frac{1}{36}$ & $\frac{2}{36}$ & $\frac{1}{36}$ & $\frac{3}{36}$ & ... \\ 
\hline 
\end{tabular} 

also: 
$EZ = 1\cdot \frac{1}{36} + 2\cdot \frac{1}{36} + 3\cdot \frac{1}{36} + 4\cdot \frac{3}{36} + 5\cdot \frac{1}{36} + 6\cdot \frac{2}{36} + 8\cdot \frac{1}{36} + 9\cdot \frac{3}{36} + ...$\\

\textbf{Mit Verfahren 1)}\\
\begin{tabular}{|c|c|c|c|}
\hline $\omega$ & X & Y & $X\cdot Y$ \\ 
\hline (1,1) & 1 & 1 & 1 \\ 
\hline (1,2) & 1 & 2 & 2 \\ 
\hline (1,3) & 1 & 3 & 3 \\ 
\hline (1,4) & 1 & 4 & 4 \\ 
\hline (1,5) & 1 & 5 & 5 \\ 
\hline (1,6) & 1 & 6 & 6 \\ 
\hline (2,1) & 2 & 2 & 4 \\ 
\hline (2,2) & 2 & 2 & 4 \\ 
\hline $\vdots$ & $\vdots$ & $\vdots$ & $\vdots$ \\ 
\hline 
\end{tabular} 
$P(\omega)=\frac{1}{36}$, 
$\frac{1}{36}(1+2+3+4+5+6+4+4+...)$

\subsection{Unabhängigkeit von 2 Zufallsvariablen}
X,Y heißen unabhängig, wenn für alle u,v gilt: 
$P(X=u, Y=v) $
$=P(X=u)\cdot P(Y=v)$

\subsubsection{Beispiel}
\begin{tabular}{|c|c|c|c|c|c|}
\hline X/Y & 1 & 2 & 3 & 4 & $\sum$\\ 
\hline 1 & $\frac{1}{6}$ & $\frac{1}{12}$ & $\frac{1}{24}$ & $\frac{1}{24}$ & $\frac{1}{3}$ \\ 
\hline 2 & $\frac{1}{6}$ & $\frac{1}{12}$ & $\frac{1}{24}$ & $\frac{1}{24}$ & $\frac{1}{3}$ \\ 
\hline 3 & $\frac{1}{6}$ & $\frac{1}{12}$ & $\frac{1}{24}$ & $\frac{1}{24}$ & $\frac{1}{3}$ \\ 
\hline $\sum$ & $\frac{1}{2}$ & $\frac{1}{4}$ & $\frac{1}{8}$ & $\frac{1}{8}$ & 1 \\ 
\hline 
\end{tabular} 

Für unabhängige ZV X und Y gilt dann:

$P(X\in A, Y\in B)$
$=P(X\in A) \cdot P(Y\in B)$

ohne Beweis: 

$A=\cbr{1,2}, B=\cbr{3,4}$
$P(X\in A) = \frac{2}{3}$,
$P(Y\in B) = \frac{1}{4}$

$P(X\in A)\cdot P(Y\in B) = \frac{2}{3} \cdot \frac{1}{4} = \frac{2}{12}$ 
$\Leftrightarrow$
$P(X\in A, Y\in B) = \frac{1}{24} + \frac{1}{24} + \frac{1}{24} + \frac{1}{24} = \frac{4}{24} $

Multiplikationsregel: Es seien X,Y \underline{unabhängige} ZV.

$E(X\cdot Y) = EX\cdot EY$, 
$E(X\cdot Y) = \sum_{i=1}^{r} \sum_{j=1}^{s} x_i y_j \cdot P(X=x_i, Y=y_j)$ \profnote{$x_i, y_i$ stehen in der Tabelle.}
$=\sum_{i=1}^{r} \sum_{j=1}^{s} x_i y_j \cdot P(X=x_i) \cdot P(Y=y_j)$
$=[\sum_{i=1}^{r} x_i \cdot P(X=x_i)] $
$\cdot $
$[\sum_{j=1}^{s} y_j\cdot P(Y=y_j)]$
$= EX \cdot EY$\\

\textbf{Immer gilt:} $E(X+Y)= EX + EY$