% Vorlesung vom 02.11.2015
\renewcommand{\ldate}{2015-11-02}

\section{Erwartungswert}
\includegraphicsdeluxe{gluecksrad1.jpg}{Glücksrad}{Glücksrad mit $\Omega = \cbr{\omega_1, \omega_1, ..., \omega_s} $}{fig:gluecksrad1}
Glücksrad mit $\Omega = \cbr{\omega_1, \omega_1, ..., \omega_s} $ (Abb. \ref{fig:gluecksrad1}). Gegeben ist $P(\cbr{\omega_1})$. Bei $ \omega_i $ erhält man den Gewinn $ X(\omega_i) $. Wie groß ist der durchschnittliche Gewinn? Wir drehen n mal: \\
$h_1$ mal $\omega_1$, $h_1 + h_2 + ... + h_s = n$\\
$h_2$ mal $\omega_2$\\
$\vdots$\\
$h_s$ mal $\omega_s$\\
Gesamtgewinn: $\sum_{j=1}^{s} X(\omega_j) \cdot h_j$\\
Durchschnittsgewinn: $\sum_{j=1}^{n} X(\omega_j) \cdot \underbrace{\frac{h_j}{n}}_{\textrm{relative Häufigkeit von } \omega_j \approx P(\cbr{\omega_j})  }$

\begin{defi}
$ X : \Omega \rightarrow \R $ ist Zufallsgröße.
Erwartungswert von 
$ X = E X $
$= \sum_{\omega \in \Omega} X(\omega) \cdot P(\cbr{\omega})$
\end{defi}

\subsection{Beispiel Würfeln}
$\Omega = \cbr{1,2,3,4,5,6}, X : \omega \rightarrow \omega^2$\\
$E X = 1^2 \cdot \frac{1}{6} $
$+ 2^2 \cdot \frac{1}{6} $
$+ 3^2 \cdot \frac{1}{6} $
$+ 4^2 \cdot \frac{1}{6} $
$+ 5^2 \cdot \frac{1}{6} $
$+ 6^2 \cdot \frac{1}{6} $
$= 15.17$

\subsection{Andere Berechnung des Erwartungswertes}
$ X : \Omega \rightarrow \R, \Omega =\cbr{\omega_1, ..., \omega_n} $. Der Wertebereich von X sei: $\cbr{x_1, ..., x_s}$.\\
$E X = \sum_{i=1}^{s} x_i \cdot P(X=x_i) $
$=\sum_{i=1}^{s} x_i \cdot P(\cbr{\omega \in \Omega : X(\omega) = x_i}) $
$= \sum_{i=1}^{s} x_i \cdot \sum_{X(\omega_j)=x_i} P(\cbr{\omega_j}) $
$= \sum_{i=1}^{s} \sum_{X(\omega_j)=x_i} x_i \cdot P(\cbr{\omega_j}) $
$=\sum_{j=1}^{n} X(\omega_j) \cdot P(\cbr{\omega_j})$

\subsection{Beispiel Würfeln mit zwei Würfeln}
$ X(\omega_1, \omega_2) = max\cbr{\omega_1, \omega_2}$\\
$ E X = $
$ 1 \cdot \frac{1}{36}$
$ + 2 \cdot \frac{3}{36}$ 
$ + 3 \cdot \frac{5}{36}$ 
$ + 4 \cdot \frac{7}{36}$ 
$ + 5 \cdot \frac{9}{36}$ 
$ + 6 \cdot \frac{11}{36}$ 
$= 4.47 $

\subsection{Satz}
\profnote{Wichtiger Satz!}
\begin{satz}
$X, Y : \Omega \rightarrow \R $ Zufallsgrößen, $ A \subset \Omega$

\begin{enumerate}
\item $ E(X+Y) = E X + E Y $
\item $ E(a \cdot X) = a \cdot E X$
\item $ E(I_A) = P(A) $
\item Aus $X \leq Y$ folgt $ E X \leq E Y $ $ \sbr{X(\omega) \leq Y(\omega), \forall \omega \in \Omega} $
\end{enumerate}
\end{satz}

\begin{proof} des o.g. Satzes. 
\begin{enumerate}
\item $ E(X+Y) = \sum_{\omega \in \Omega} (X+Y)(\omega) \cdot p(\omega)$
$= \sum_{\omega} X(\omega) \cdot p(\omega) $
$+ \sum_{\omega} Y(\omega) \cdot p(\omega) $
$ = E X + E Y $
\item analog 1.
\item $E(I_A) $
$=\sum_{\omega} I_A(\omega) \cdot p(\omega) $
$= \sum_{\omega \in A} 1 \cdot p(\omega)$
$= P(A)$
\item klar.
\end{enumerate}
\end{proof}
Es folgt: $ E(X_1 + X_2 + ... + X_n) $
$ = E X_1 + E X_2 + ... + E X_n $
Es seien $A_1, A_2, ..., A_n$ Ereignisse. Zählvariable ist X: \\
$X(\omega) = $ Anzahl der $A_i$ in denen $\omega$ liegt. $ X = I_{A_1} + I_{A_2} + ... + I_{A_n}$

\subsection{Beispiel Rekorde}
$\Omega: $ Permutationen der Zahlen 1 bis n $(a_1,a_2,...,a_j,...,a_n)$ \profnote{$a_j$ ist ein Rekord, falls alle vorderen kleiner sind.}\\
$X(\omega) = $ Anzahl der Rokorde von $\omega$, $A_j: \cbr{\rbr{a_1, ..., a_n} : a_j \textrm{ ist Rekord}}$, $\Rightarrow X = I_{A_1} + I_{A_2} + ... + I_{A_n}$\\

Wir berechnen $P(A_j)$: 
$\cbr{\rbr{a_1,a_2,...,a_j}}, 1 \leq a_s \leq n $ \profnote{Wir brechen das Tupel bei $a_j$ ab.}

alle: $\binom n j \cdot j!$

günstige: $\binom n j \cdot \rbr{j-1}!$

Wahrscheinlichkeit von $A_j$ (Abzählformel): $ P(A_j) = \frac{\binom n j \cdot \rbr{j-1}!}{\binom n j \cdot j!}$
$= \frac{1}{j}$

Also: $ E X = $
$E I_{A_1} + E I_{A_2} + ... + E I_{A_n}$
$= P(A_1) + P(A_2) + ... + P(A_n)$
$= 1 + \frac{1}{2} + \frac{1}{3} + ... + \frac{1}{n} $\\


z.B.: $n=7$ Permutationen der Zahlen 1 bis 7. 
$\Rightarrow 1$
$+ \frac{1}{2}$
$+ \frac{1}{3}$ 
$+ \frac{1}{4}$ 
$+ \frac{1}{5}$ 
$+ \frac{1}{6}$
$=2.6$

\subsection{Näherungsformel}
\includegraphicsdeluxe{einsDurchX1.jpg}{Wir berechnen die Flächen der roten Rechtecke unterhalb der Funktion $f(x) = \frac{1}{x}$}{$f(x) = \frac{1}{x}$}{fig:einsDurchX1}
Wir berechnen die Flächen der Rechtecke (Abb. \ref{fig:einsDurchX1}). 
$ \frac{1}{2} + \frac{1}{3} + ... + \frac{1}{n} $
$ \leq \int_{1}^{n} \frac{1}{x} dx $ 
$ = \sbr{\ln x}_1^n $ 
$ = \ln n - \ln 1 $
$ = \ln n $

Permutationen der Zahlen 1 bis n. X: Anzahl der Rekorde 
$\Rightarrow E X \leq 1 + \ln n $

z.B.: $n=100 \Rightarrow E X = 5.6$\\
$n=1 000 000\Rightarrow E X = 14.8$
