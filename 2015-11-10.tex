% Vorlesung vom 10.11.2015
\renewcommand{\ldate}{2015-11-10}

\subsection{Beispiel}
Wir haben $r=1, s=3, c=1$ Kugeln. Wir ziehen $n=4$ mal. Gesucht ist die Wahrscheinlichkeit, dass ich genau zwei rote Kugeln hab, also: $P(X=2)$.
\profnote{c sind die Kugeln, die wir wieder zurücklegen. }
$ \prod_{j=0}^{n-1} (r+s+j\cdot c)$
$=\prod_{j=0}^{3} (5+j)$
$=5\cdot 6\cdot 7\cdot 8$
$=1680$

$\prod_{j=0}^{k-1} (r+j\cdot c)$
$=\prod_{j=0}^{1} (2+J)$
$=2\cdot 3$
$=6$

$\prod_{j=0}^{n-k-1} (s+j\cdot c)$
$=\prod_{j=0}^{1} (3+j)$
$=3\cdot 4$
$=12$

$\binom n k $
$=\binom 4 2$
$=\frac{4\cdot 3}{1\cdot 2}$
$=6$

$P(X=2)$
$=6 \cdot \frac{6\cdot 12}{1680}$
$=0.257$\\

Wir wollen den Erwartungswert von X berechnen, also wie viele rote Kugeln durchschnittlich gezogen werden. Dazu schauen wir uns das Ereignis $A_j$ an (Menge aller Ziehungen), also: $A_j=\cbr{(a_1,a_2,...,a_n) : a_j=1}$, 
$P(A_1) = \frac{r}{r+s}$ (klar). Es gilt aber für jedes j: 
$P(A_j) = \frac{r}{r+s}$, z.B. $n=3$: 
$P(A_1) = p(\textbf{1},0,0) + p(\textbf{1},0,1) + p(\textbf{1},1,0) + p(\textbf{1},1,1)$
$=p(0,\textbf{1},0) + p(0,\textbf{1},1) + p(1,\textbf{1},0) + p(1,\textbf{1},1) $
$=P(A_2)$

\textbf{Es gilt sogar:} Die Ereignisse $A_1, ..., A_n$ sind austauschbar, d.h. 
$P(A_1\cap A_2\cap ... \cap A_n)$
$=P(A_{i_1}\cap A_{i_2}\cap ... \cap A_{i_n}$, 
z.B.$n=4$: 
$P(A_1\cap A_2)$
$=p(\textbf{1},\textbf{1},0,0) + p(\textbf{1},\textbf{1},0,1) + p(\textbf{1},\textbf{1},1,0) + p(\textbf{1},\textbf{1},1,1) $
$=p(0,0,\textbf{1},\textbf{1}) + p(1,0,\textbf{1},\textbf{1}) + p(0,1,\textbf{1},\textbf{1}) + p(1,1,\textbf{1},\textbf{1}) $
$=P(A_3\cap A_4)$\\

Also: $X=I_{A_1} + I_{A_2} + ... + I_{A_n} $,
$ E X = \sum_{j=1}^{n} E(I_{A_j})$
$ = \sum_{j=1}^{n} P(A_j)$
$= \sum_{j=1}^{n} \frac{r}{r+s}$
$=\underline{n \frac{r}{r+s}} $
$= E X $

In unserem Beispiel: $r=1, s=3, c=1, n=4$, 
$ E X = 4\cdot \frac{2}{2+3}$
$=\frac{8}{5}$
$=1.6$ (rote Kugeln im Durchschnitt) 

\subsection{Beispiel}
\includegraphicsdeluxe{BspSchachteln1.jpg}{Beispiel Schachteln}{Beispiel Schachteln: $ w+s=c, c\leq 100 $}{fig:BspSchachteln1}
100 weiße und 100 schwarze Kugeln werden auf 2 Schachteln (keine leer) verteilt. Schachteln wählen, Kugeln ziehen. Man gewinnt, wenn die Kugel weiß ist. 
Idee: Wir legen eine weiße Kugel in eine Schachtel und die restlichen 99 weißen Kugeln und die 100 schwarzen in die andere Schachtel (Abb. \ref{fig:BspSchachteln1}).

P(weiße Kugeln)
$=\frac{1}{2} \cdot \frac{w}{c} + \frac{1}{2} \cdot \frac{100-w}{200-c}$, 
\textbf{maximiere:} $\frac{w}{c} + \frac{100-w}{200-c}$
$=f(w) | c $ Konstante. \\

$f(w) = \frac{1}{c} \cdot w + \frac{100}{200-c} - \frac{1}{200-c} \cdot w$
$=w \underbrace{\rbr{\frac{1}{c} - \frac{1}{200-c}}}_{> 0} + \frac{100}{200-c}$ (Geradengleichung)
$\Rightarrow$ w möglichst groß wählen, also $w=c$.

\textbf{maximiere:} $\underbrace{\frac{c}{c}}_{1} + \underbrace{\frac{100-c}{200-c}}_{\textrm{maximal}}$

\textbf{minimiere:} $\frac{200-c}{100-c}$
$\frac{100-c+100}{100-c}$
$=\frac{100-c}{100-c}$
$+ \frac{100}{100-c}$
$=1 + \frac{100}{100-c}$ minimal bei $c=1$. 

\textbf{Optimal:} $c=1, w=1$ in eine Schachtel eine weiße Kugel. 
P(weiße Kugel) $=\frac{1}{2} \cdot \frac{1}{1} + \frac{1}{2} \cdot \frac{100-1}{200-1}$
$=\frac{1}{2} + \frac{99}{2\cdot 199}$
$\approx 0.75$ 