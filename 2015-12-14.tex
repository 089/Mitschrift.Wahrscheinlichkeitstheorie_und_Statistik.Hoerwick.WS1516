% Vorlesung vom 14.12.2015
\renewcommand{\ldate}{2015-12-14}

\subsection{Beispiel hypergeometrische Verteiltung}
Wir haben r rote und s schwarze Kugeln in einer Urne. Es werden n Kugeln ohne Rücklegen gezogen: X ist die Anzahl der roten Kugeln, $A_j$ das Ereignis: bei jeder j-ten Ziehung rot $(a_1, a_s, ..., \underbrace{a_j}_{\textrm{hier rot}}, ..., a_n)$

$X = I_{A_1} + ... + I_{A_n}$

$P(A_j)$
$=\frac{r}{r+s}$, 
$P(A_i \cap A_j)$
$=\frac{r(r-1)}{(r+s)(r+s-1)}$

$V(I_{A_1} + ... + I_{A_n}) $
$=n\cdot \frac{r}{r+s} \cdot \rbr{1-\frac{r}{r+s}} + n(n-1) \sbr{\frac{r(r-1)}{(r+s)(r+s-1)} - \rbr{\frac{r}{r+s}}^2}$
$=...$
$=n\cdot \frac{r}{r+s} \cdot \rbr{1-\frac{r}{r+s}} \rbr{1-\frac{n-1}{r+s-1}}$

Mit $p=\frac{r}{r+s}$ 

$V(X) = n\cdot p(1-p) \rbr{1- \frac{n-1}{r+s-1}}$

$[\sbr{EX=n \cdot p}]$

\textbf{Vergleich zur Binomialverteilung} $EX = np, V(X)=np(1-p)$

\subsection{Beispiel Permutationen}
Betrachte die Permutationen der Zahlen 1 bis n. X ist die Anzahl der Fixpunkte einer Permutation.

$A_j: \cbr{\textrm{Permutationen }\omega : \textrm{Fixpunkte bei j}}$ 

$ X = I_{A_1} + I_{A_2} + ... + I_{A_n}$

$P(A_j) = \frac{1}{n}$

$P(A_i\cap A_j) = \frac{1}{n(n-1)}$, auch möglich: $= \frac{(n-2)!}{n!}$

$ V(X) $
$=n P(A_1)(1-P(A_1))+ n(n-1) \sbr{P(A_1\cap A_2) - P(A_1)^2}$
$=\rbr{1-\frac{1}{n}} + (n-1) \sbr{\frac{1}{n-1} - \frac{1}{n}}$
$=\frac{n-1}{n} + \frac{n-1}{n-1} - \frac{n-1}{n}$
$= 1$

$Var(X) = 1, $ von früher: $EX =1 \Rightarrow $ unabhängig von n.

\subsection{Korrelationskoeffizient}

\begin{defi}
$X,Y: \Omega \rightarrow \R$

$r(X,Y) = \frac{C(X,Y)}{\sqrt{V(X) \cdot V(Y)}}$ heißt Korrelationskoeffizient von X und Y
\end{defi}

\textbf{Problem:}
$X,Y: \Omega \rightarrow \R$. Wir kennen $X(\omega)$ und wollen $Y(\omega)$ schätzen. $Y \approx g(X), Y \approx a+bX$ (Gerade). Wie berechnet man a und b, damit man Y möglichst gut schätzt? 

Der Schätzfehler $Y(\omega) - a - b X(\omega)$ ist eine Zufallsgröße.

Minimiere $E\sbr{\rbr{Y(\omega) - a - b X(\omega)}^2}$. Dieses Optimierungsproblem hat die Lösung:
$b^* = \frac{C(X,Y)}{V(X)}, a^+ = EXY - b^* EX$.

Der Minimalwert $M^+$ von $*$ ist:
$M^+ = V(Y) \cdot \sbr{1-r^2(X,Y)}$

\textbf{Beweis:} $Z=Y-bX;$
$ E\sbr{(Y-a-bX)^2}$
$=\underbrace{E\sbr{(Z-a)^2}}$

$=\sbr{V(X) = \underbrace{E\sbr{(X-a)^2}} - \sbr{EX-a}^2}$
$=Var(Z) + \sbr{EZ-a}^2$

\textbf{minimal bei:} $EZ =a; E(Y-bX) = a; EY -bEX= a$

\profnote{Abkürzung: $\tilde{Y} = Y-EY, \tilde{X} = X-EX$}
\textbf{minimiere:} $V(Z) = V(Y-bX)$
$=E\sbr{(Y-bX - EY + bEX)^2}$
$=E \sbr{(\tilde{Y} - b\tilde{X})^2}$
$=E\sbr{\tilde{Y}^2 - 2b \tilde{X}\tilde{Y} + b^2 \tilde{X}^2}$
$=V(Y) -2b C(X,Y) + b^2 V(X) $
$=h(b)$
$\Rightarrow$ \textbf{minimal beim Scheitel:} 
$ h'(b) = -2 C(X,Y) + 2b V(X) = 0$

$b V(X) = C(X,Y)$

$b = \frac{C(X,Y)}{V(X)}$

\textbf{Also: } $b^* = \frac{C(X,Y)}{V(X)}, a^* = EY - b^* EX$\\

$M^* = E\sbr{ (Y-a^* - b^* X)^2 }$
$= E \sbr{ (Y-EY + b^* EX - b^* X)^2 }$
$= E \sbr{ ((Y-EY) - b^* (X - EX))^2 }$
$= E \sbr{ ((Y-EY)^2 - b^*{^2} (X - EX)^2 - 2b^* (x-EX) (Y-EY) }$
$= V(Y) + b^* V(X) - 2 b^* C(X,Y)$
$=V(Y) + \frac{C(X,Y)^2 \cdot V(X)}{V(X)^2} - 2 \cdot \frac{C(X,Y)}{V(X)} \cdot C(X,Y)$
$=V(Y) + \frac{C(X,Y)^2}{V(X)} - 2 \cdot \frac{C(X,Y)^2}{V(X)}$
$=V(Y) - \frac{C(X,Y)^2}{V(X)}$
$=V(Y) - V(Y) \cdot \frac{C(X,Y)^2}{V(X)\cdot V(Y)}$
$=V(Y) \cdot \sbr{1 - r(X,Y)^2}$

\subsection{Folgerungen}

\begin{enumerate}
\item $\sbr{C(X,Y)}^2 \leq V(X) \cdot V(Y)$
\item $\abs{r(X,Y)} \leq 1$
\item $\abs{r(X,Y)} = 1 \Leftrightarrow M^* = 0 \Leftrightarrow Y = a+bX$
\end{enumerate}