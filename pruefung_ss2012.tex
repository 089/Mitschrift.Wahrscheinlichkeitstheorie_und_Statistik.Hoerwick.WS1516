% \renewcommand{\ldate}{2016-01-}
\includepdf[pages=-]{pruefungsangabe_wahrStat_ss2012}

\section{Lösung für die Prüfung SS 2012}

\subsection{zu 1)}
%1 \includegraphicsdeluxe{baumZuSs20121.jpg}{Wahrscheinlichkeitsbaum}{Wahrscheinlichkeitsbaum}{fig:baumZuSs20121}

\subsubsection{zu 1a)}
A: Person krank\\
B: Test zeigt positiv
$P(A|B) = \frac{P\cap B}{P(B)} = \frac{0.01\cdot 0.99}{0.01\cdot 0.99 + 0.99\cdot 0.10} = 0.09$

\subsubsection{zu 1b)}
Jetzt ist die apriori Wahrscheinlichkeit, dass er krank ist, höher (vielleicht 0.1 statt 0.01). Damit ist die aposteriori Wahrscheinlichkeit, dass er krank ist, auch höher (vielleicht 0.5). 

\subsection{zu 2a)}

\begin{tabular}{|c|c|c|c|c|}
\hline X/Y & 1 & 4 & 7 &  \\ 
\hline 1 & 0.2 & 0.1 & 0.05 & 0.35 \\ 
\hline 2 & 0.05 & 0.2 & 0.05 & 0.3 \\ 
\hline 3 & 0.05 & 0.1 & 0.2 & 0.35 \\ 
\hline  & 0.3 & 0.4 & 0.3 &  \\ 
\hline 
\end{tabular} 

\begin{enumerate}
\item Immer multiplizieren Zeile mal Randwahrscheinlichkeit: $ EX = 1\cdot 0.35 + 2\cdot 0.3 + 3\cdot 0.35 = 2$
\item Wert mal Wahrscheinlichkeit: $ 1\cdot 0.3 + 4\cdot 0.4 + 7\cdot 0.3 = 4$
\item $EX^2 = 1^2 \cdot 0.35 + 2^2\cdot 0.3 + 3^2\cdot 0.35 = 1\cdot 0.35 + 4\cdot 0.3 + 9\cdot 0.35 = 4.7$\\
$EY^2 = 1\cdot 0.3 + 16\cdot 0.4 + 49\cdot 0.3 = 21.4 $\\
$Var(X) = E(X^2) - (EX)^2 = 4.7 - 4 = 0.7$\\
$Var(Y) = E(Y^2) - (EY)^2 = 21.4 - 16 = 5.4$\\
\item Nicht einfach multiplizieren. Ganze Tabelle durchmachen: $ E(X\cdot Y) = 
1\cdot 1\cdot 0.2 + 
1\cdot 4\cdot 0.1 + 
1\cdot 7\cdot 0.05 + 
2\cdot 1\cdot 0.05 + 
2\cdot 4\cdot 0.2 + 
2\cdot 7\cdot 0.05 + 
3\cdot 1\cdot 0.05 + 
3\cdot 4\cdot 0.1 + 
3\cdot 7\cdot 0.2 = 8.9
$
\item $ Cov(X,Y) = E(X\cdot Y) - EX\cdot EX = 8.9 - 2 \cdot 4 = 0.9$
\end{enumerate}

\subsection{zu 2b)}
$ E(7X - 3Y + 5) = 7 EX - 3 EY + 5 = 7\cdot 2 - 3\cdot 4 + 5 = 7$